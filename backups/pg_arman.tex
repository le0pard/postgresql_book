\subsection{Pg\_arman}

\href{https://github.com/michaelpq/pg\_arman}{Pg\_arman}~--- менеджер резервного копирования и восстановления для PostgreSQL 9.5 или выше. Это ответвление проекта \lstinline!pg_arman!, изначально разрабатываемого в NTT. Теперь его разрабатывает и поддерживает Мишель Пакье. Утилита предоставляет следующие возможности:

\begin{itemize}
  \item Резервное копирование во время работы базы данных, включая табличные пространства, с помощью всего одной команды;
  \item Восстановление из резервной копии всего одной командой, с нестандартными вариантами, включая использование \href{https://www.postgresql.org/docs/current/static/continuous-archiving.html}{PITR};
  \item Поддержка полного и дифференциального копирования;
  \item Управление резервными копиями со встроенными каталогами;
\end{itemize}


\subsubsection{Использование}

Сначала требуется создать <<каталог резервного копирования>>, в котором будут храниться файлы копий и их метаданные. До инициализации этого каталога рекомендуется настроить параметры \lstinline!archive_mode! и \lstinline!archive_command! в \lstinline!postgresql.conf!. Если переменные инициализированы, \lstinline!pg_arman! может скорректировать файл конфигурации. В этом случае потребуется задать путь к кластеру баз данных: переменной окружения \lstinline!PGDATA! или через параметр \lstinline!-D/--pgdata!.

\begin{lstlisting}[language=Bash,label=lst:pgarman1,caption=init]
$ pg_arman init -B /path/to/backup/
\end{lstlisting}

После этого возможен один из следующих вариантов резервного копирования:

\begin{itemize}
  \item Полное резервное копирование (копируется весь кластер баз данных);
  \begin{lstlisting}[language=Bash,label=lst:pgarman2,caption=backup]
  $ pg_arman backup --backup-mode=full
  $ pg_arman validate
  \end{lstlisting}
  \item Дифференциальное резервное копирование: копируются только файлы или страницы, изменённые после последней проверенной копии. Для этого выполняется сканирование записей WAL от позиции последнего копирования до LSN выполнения \lstinline!pg_start_backup! и все изменённые блоки записываются и отслеживаются как часть резервной копии. Так как просканированные сегменты WAL должны находиться в архиве WAL, последний сегмент, задействованный после запуска \lstinline!pg_start_backup!, должен быть переключен принудительно;
  \begin{lstlisting}[language=Bash,label=lst:pgarman3,caption=backup]
  $ pg_arman backup --backup-mode=page
  $ pg_arman validate
  \end{lstlisting}
\end{itemize}

После резервного копирования рекомендуется проверять файлы копий как только это будет возможно. Непроверенные копии нельзя использовать в операциях восстановления и резервного копирования.

До начала восстановления через \lstinline!pg_arman! PostgreSQL кластер должен быть остановлен. Если кластер баз данных всё ещё существует, команда восстановления сохранит незаархивированный журнал транзакций и удалит все файлы баз данных. После восстановления файлов \lstinline!pg_arman! создаёт \lstinline!recovery.conf! в \lstinline!$PGDATA! каталоге. Этот конфигурационный файл содержит параметры для восстановления. После успешного восстановления рекомендуется при первой же возможности сделать полную резервную копию. Если ключ \lstinline!--recovery-target-timeline! не задан, целевой точкой восстановления будет \lstinline!TimeLineID! последней контрольной точки в файле (\lstinline!$PGDATA/global/pg_control!). Если файл \lstinline!pg_control! отсутствует, целевой точкой будет \lstinline!TimeLineID! в полной резервной копии, используемой при восстановлении.

\begin{lstlisting}[language=Bash,label=lst:pgarman4,caption=restore]
$ pg_ctl stop -m immediate
$ pg_arman restore
$ pg_ctl start
\end{lstlisting}

\lstinline!Pg_arman! имеет ряд ограничений:

\begin{itemize}
  \item Требуется права чтения каталога баз данных и записи в каталог резервного копирования. Обычно для этого на сервере БД требуется смонтировать диск, где размещён каталог резервных копий, используя NFS или другую технологию;
  \item Основные версии \lstinline!pg_arman! и сервера должны совпадать;
  \item Размеры блоков \lstinline!pg_arman! и сервера должны совпадать;
  \item Если в каталоге с журналами сервера или каталоге с архивом WAL оказываются нечитаемые файлы/каталоги, резервное копирование или восстановление завершится сбоем, вне зависимости от выбранного режима копирования;
\end{itemize}
