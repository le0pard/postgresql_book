\section{Pglogical}
\label{sec:pglogical}

\href{https://2ndquadrant.com/en/resources/pglogical/}{Pglogical}~--- это расширение для PostgreSQL, которое использует логическое декодирование через publish/subscribe модель. Данное расширение базируется на BDR проекте (\ref{sec:bdr}~\nameref{sec:bdr}). Расширение работает только начиная с версии PostgreSQL 9.4 и выше (из-за логического декодирования). Для разных вариаций обнаружения и разрешения конфликтов требуется версия 9.5 и выше.

Используются следующие термины для описания pglogical:

\begin{itemize}
  \item Nodes(ноды, узлы)~--- экземпляры баз данных PostgreSQL;
  \item Provider и subscriber~--- роли узлов. Provider выполняют выдачу данных и изменений для subscriber-ов;
  \item Replication set (набор для репликации)~--- коллекция таблиц и последовательностей для репликации;
\end{itemize}

Сценарии использования pglogical:

\begin{itemize}
  \item Обновление между версиями PostgreSQL (например c 9.4 на 9.5);
  \item Полная репликация базы данных;
  \item Выборочная репликация таблиц;
  \item Сбор данных с нескольких баз данных в одну;
\end{itemize}

Архитектурные детали:

\begin{itemize}
  \item Pglogical работает на уровне каждой базы данных, а не на весь сервер;
  \item Provider (publisher) может <<кормить>> несколько subscriber-ов без дополнительных накладных расходов записи на диск;
  \item Один subscriber может объединить изменения из нескольких provider-ов и использовать систему обнаружения и разрешения конфликтов между изменениями;
  \item Каскадная репликация осуществляется в виде переадресации изменений;
\end{itemize}


\subsection{Установка и настройка}


Установить pglogical можно по \href{https://2ndquadrant.com/en/resources/pglogical/pglogical-installation-instructions/}{данной документации}. Далее требуется настроить логический декодинг в PostgreSQL:

\begin{lstlisting}[label=lst:pglogical1,caption=postgresql.conf]
wal_level = 'logical'
max_worker_processes = 10   # one per database needed on provider node
                            # one per node needed on subscriber node
max_replication_slots = 10  # one per node needed on provider node
max_wal_senders = 10        # one per node needed on provider node
shared_preload_libraries = 'pglogical'
\end{lstlisting}

Если используется PostgreSQL 9.5+ и требуются механизмы разрешения конфликтов, то требуется добавить дополнительные опции:

\begin{lstlisting}[label=lst:pglogical2,caption=postgresql.conf]
track_commit_timestamp = on # needed for last/first update wins conflict resolution
                            # property available in PostgreSQL 9.5+
\end{lstlisting}

В \lstinline!pg_hba.conf! нужно разрешить replication соеденения с локального хоста для пользователя с привилегией репликации. После перезапуска базы нужно активировать расширение на всех нодах:

\begin{lstlisting}[label=lst:pglogical3,language=SQL,caption=Активируем расширение]
CREATE EXTENSION pglogical;
\end{lstlisting}

Далее на master (мастер) создаем provider (процесс, который будет выдавать изменения для subscriber-ов) ноду:

\begin{lstlisting}[label=lst:pglogical4,language=SQL,caption=Создаем provider]
SELECT pglogical.create_node(
    node_name := 'provider1',
    dsn := 'host=providerhost port=5432 dbname=db'
);
\end{lstlisting}

И добавляем все таблицы в \lstinline!public! схеме:

\begin{lstlisting}[label=lst:pglogical5,language=SQL,caption=Добавляем в replication set все таблицы в public схеме]
SELECT pglogical.replication_set_add_all_tables('default', ARRAY['public']);
\end{lstlisting}

Далее переходим на slave (слейв) и создаем subscriber ноду:

\begin{lstlisting}[label=lst:pglogical6,language=SQL,caption=Создаем subscriber]
SELECT pglogical.create_node(
    node_name := 'subscriber1',
    dsn := 'host=thishost port=5432 dbname=db'
);
\end{lstlisting}

После этого создаем <<подписку>> на provider ноду, которая начнет синхронизацию и репликацию в фоне:

\begin{lstlisting}[label=lst:pglogical7,language=SQL,caption=Активируем subscriber]
SELECT pglogical.create_subscription(
    subscription_name := 'subscription1',
    provider_dsn := 'host=providerhost port=5432 dbname=db'
);
\end{lstlisting}

Если все проделано верно, subscriber через определенный интервал времени subscriber нода должна получить точную копию всех таблиц в \lstinline!public! схеме с master хоста.


\subsection{Разрешение конфликтов}


Если используется схема, где subscriber нода подписана на данные из нескольких provider-ов, или же на subscriber дополнительно производятся локальные изменения данных, могут возникать конфликты для новых изменений. В pglogical встроен механизм для обнаружения и разрешения конфликтов. Настройка данного механизма происходит через  \lstinline!pglogical.conflict_resolution! ключ. Поддерживаются следующие значения:

\begin{itemize}
  \item \lstinline!error! - репликация остановится на ошибке, если обнаруживается конфликт и потребуется ручное действие для его разрешения;
  \item \lstinline!apply_remote! - всегда применить изменения, который конфликтуют с локальными данными. Значение по умолчанию;
  \item \lstinline!keep_local! - сохранить локальную версию данных и игнорировать конфликтующие изменения, которые исходят от provider-а;
  \item \lstinline!last_update_wins! - версия данных с самым новым коммитом (newest commit timestamp) будет сохранена;
  \item \lstinline!first_update_wins! - версия данных с самым старым коммитом (oldest commit timestamp) будет сохранена;
\end{itemize}

Когда опция \lstinline!track_commit_timestamp! отключена, единственное допустимое значение для \lstinline!pglogical.conflict_resolution! может быть \lstinline!apply_remote!. Поскольку \lstinline!track_commit_timestamp! не доступен в PostgreSQL 9.4, данная опция установлена по умолчанию в \lstinline!apply_remote!.



\subsection{Ограничения и недостатки}

\begin{itemize}
  \item Для работы требуется суперпользователь;
  \item \lstinline!UNLOGGED! и \lstinline!TEMPORARY! таблицы не реплицируются;
  \item Для каждой базы данных нужно настроить отдельный provider и subscriber;
  \item Требуется primary key или \href{http://www.postgresql.org/docs/current/static/sql-altertable.html#SQL-CREATETABLE-REPLICA-IDENTITY}{replica identity} для репликации;
  \item Разрешен только один уникальный индекс/ограничение/основной ключ на таблицу (из-за возможных конфликтов). Возможно использовать больше, но только в случае если subscriber читает только с одного provider и не производится локальных изменений данных на нем;
  \item Автоматическая репликация DDL не поддерживается. У pglogical есть команда \lstinline!pglogical.replicate_ddl_command! для запуска DDL на provider и subscriber;
  \item Ограничения на foreign ключи не выполняются на subscriber-рах;
  \item При использовании \lstinline!TRUNCATE ... CASCADE! будет выполнен \lstinline!CASCADE! только на provider;
  \item Последовательности реплицируются периодически, а не в режиме реального времени;
\end{itemize}
