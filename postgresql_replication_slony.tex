\section{Slony-I}
\subsection{Введение}
Slony это система репликации реального времени, позволяющая организовать синхронизацию нескольких серверов 
PostgreSQL по сети. Slony использует триггеры Postgre для привязки к событиям INSERT/ DELETE/UPDATE и 
хранимые процедуры для выполнения действий.

Система Slony с точки зрения администратора состоит из двух главных компонент, репликационного демона slony и 
административной консоли slonik. Администрирование системы сводится к общению со slonik-ом, демон slon только 
следит за собственно процессом репликации. А админ следит за тем, чтобы slon висел там, где ему положено. 

\subsubsection{О slonik-e}
Все команды slonik принимает на свой stdin. До начала выполнения скрипт slonik-a проверяется на соответствие синтаксису, 
если обнаруживаются ошибки, скрипт не выполняется, так что можно не волноваться если slonik сообщает о syntax error, 
ничего страшного не произошло. И он ещё ничего не сделал. Скорее всего. 

\subsection{Установка}
Установка на Ubuntu производится простой командой:
\begin{verbatim}
sudo aptitude install slony1-bin
\end{verbatim}

\subsection{Настройка}
Рассмотрим теперь установку на гипотетическую базу данных customers 
(названия узлов, кластеров и таблиц являются вымышленными).

Наши данные
\begin{itemize}
\item БД: customers
\item master\_host: customers\_master.com
\item slave\_host\_1: customers\_slave.com
\item cluster name (нужно придумать): customers\_rep
\end{itemize}

\subsubsection{Подготовка master-сервера}