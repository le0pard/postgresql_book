\chapter{Мультиплексоры соединений}

\begin{epigraphs}
\qitem{Если сразу успеха не добились, пытайтесь снова и снова.
Затем оставьте эти попытки. Какой смысл глупо упорствовать?}{Уильям Клод Филдс}
\end{epigraphs}

\section{Введение}

Мультиплексоры соединений (программы для создания пула соединений) позволяют уменьшить накладные расходы на базу данных, в случае, когда огромное количество физических соединений ведет к падению производительности PostgreSQL. Это особенно важно на Windows, когда система ограничивает большое количество соединений. Это также важно для веб-приложений, где количество соединений может быть очень большим.

Для PostgreSQL существует PgBouncer и Pgpool-II, которые работают как мультиплексоры соединений.

\section{PgBouncer}

Это мультиплексор соединений для PostgreSQL от компании Skype. Существуют три режима управления:

\begin{itemize}
  \item Session Pooling~--- наиболее <<вежливый>> режим. При начале сессии клиенту выделяется соединение с сервером; оно приписано ему в течение всей сессии и возвращается в пул только после отсоединения клиента;
  \item Transaction Pooling~--- клиент владеет соединением с бэкендом только в течение транзакции. Когда PgBouncer замечает, что транзакция завершилась, он возвращает соединение назад в пул;
  \item Statement Pooling~--- наиболее агрессивный режим. Соединение с бэкендом возвращается назад в пул сразу после завершения запроса. Транзакции с несколькими запросами в этом режиме не разрешены, так как они гарантировано будут отменены. Также не работают подготовленные выражения (prepared statements) в этом режиме;
\end{itemize}

К достоинствам PgBouncer относится:

\begin{itemize}
  \item малое потребление памяти (менее 2 КБ на соединение);
  \item отсутствие привязки к одному серверу баз данных;
  \item реконфигурация настроек без рестарта.
\end{itemize}

Базовая утилита запускается просто:

\begin{lstlisting}[language=Bash,label=lst:pgbouncer1,caption=PgBouncer]
$ pgbouncer [-d][-R][-v][-u user] <pgbouncer.ini>
\end{lstlisting}

Пример конфига:

\begin{lstlisting}[label=lst:pgbouncer2,caption=PgBouncer]
[databases]
template1 = host=127.0.0.1 port=5432 dbname=template1
[pgbouncer]
listen_port = 6543
listen_addr = 127.0.0.1
auth_type = md5
auth_file = userlist.txt
logfile = pgbouncer.log
pidfile = pgbouncer.pid
admin_users = someuser
\end{lstlisting}

Нужно создать файл пользователей \lstinline!userlist.txt! примерно такого содержания: \lstinline!"someuser" "same_password_as_in_server"!. Административный  доступ из консоли к базе данных pgbouncer можно получить через команду ниже:

\begin{lstlisting}[language=Bash,label=lst:pgbouncer3,caption=PgBouncer]
$ psql -h 127.0.0.1 -p 6543 pgbouncer
\end{lstlisting}

Здесь можно получить различную статистическую информацию с помощью команды \lstinline!SHOW!.


\section{PgPool-II vs PgBouncer}

Если сравнивать PgPool-II и PgBouncer, то PgBouncer намного лучше работает с пулами соединений, чем PgPool-II. Если вам не нужны остальные возможности, которыми владеет PgPool-II (ведь пулы коннектов это мелочи к его функционалу), то конечно лучше использовать PgBouncer.

\begin{itemize}
  \item PgBouncer потребляет меньше памяти, чем PgPool-II;
  \item у PgBouncer возможно настроить очередь соединений;
  \item в PgBouncer можно настраивать псевдо базы данных (на сервере они могут называться по-другому);
\end{itemize}

Также возможен вариант использования PgBouncer и PgPool-II совместно.
