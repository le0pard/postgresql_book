\chapter{Партиционирование}
\begin{epigraphs}
\qitem{Решая какую-либо проблему, всегда полезно заранее знать правильный ответ. 
При условии, конечно, что вы уверены в наличии самой проблемы.}{Народная мудрость}
\end{epigraphs}
\section{Введение}
Партиционирование (partitioning)~--- это разбиение больших таблиц на логические части по выбранным критериям. 
Звучит сложно, но на практике все просто.

Скорее всего у Вас есть несколько огромных таблиц (обычно всю нагрузку обеспечивают всего несколько таблиц СУБД из всех имеющихся). 
Причем чтение в большинстве случаев приходится только на самую последнюю их часть (т.е. активно читаются те данные, которые 
недавно появились). Примером тому может служить блог — на первую страницу (это последние 5\dots10 постов) приходится 40\dots50\% 
всей нагрузки, или новостной портал (суть одна и та же), или системы личных сообщений… впрочем понятно. Партиционирование 
таблицы позволяет базе данных делать интеллектуальную выборку — сначала СУБД уточнит, какой партиции соответствует Ваш запрос 
(если это реально) и только потом сделает этот запрос, применительно к нужной партиции (или нескольким партициям). Таким образом, 
в рассмотренном случае, Вы распределите нагрузку на таблицу по ее партициям. Следовательно выборка типа 
<<SELECT * FROM articles ORDER BY id DESC LIMIT 10>> будет выполняться только над последней партицией, которая значительно 
меньше всей таблицы.

Многие СУБД поддерживают партиционирование на том или ином уровне, например:
dev.mysql.com/doc/refman/5.1/en/partitioning.html — Партиционирование в Mysql, 
отлично реализовано на уровне СУБД (убедитесь, что Ваша версия >= 5.1).
www.postgresql.org/docs/8.1/interactive/ddl-partitioning.html — Партиционирование в Postgres, 
не так хорошо, но все же возможность есть.