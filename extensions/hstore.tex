\section{HStore}
\textbf{Лицензия}: Open Source

HStore~-- расширение, которое реализует тип данных для хранения ключ/значение в пределах одного значения в PostgreSQL (например, в одном текстовом поле). Это может быть полезно в различных ситуациях, таких как строки с многими атрибутами, которые редко вибираются, или полу-структурированные данные. Ключи и значения являются простыми текстовыми строками.

\subsection{Пример использования}
Для начала активируем расширение:
\begin{lstlisting}[label=lst:hstore1,caption=Активация hstore]
# CREATE EXTENSION hstore;
\end{lstlisting}

Проверим работу расширения:
\begin{lstlisting}[label=lst:hstore2,caption=Проверка hstore]
# SELECT 'a=>1,a=>2'::hstore;
  hstore
----------
 "a"=>"1"
(1 row)
\end{lstlisting}

Как видно на листинге~\ref{lst:hstore2} ключи в hstore уникальны. Создадим таблицу и заполним её данными:

\begin{lstlisting}[label=lst:hstore3,caption=Проверка hstore]
CREATE TABLE products (
   id serial PRIMARY KEY,
   name varchar,
   attributes hstore
);
INSERT INTO products (name, attributes)
VALUES (
  'Geek Love: A Novel',
  'author    => "Katherine Dunn",
  pages     => 368,
  category  => fiction'
),
(
 'Leica M9',
 'manufacturer  => Leica,
  type          => camera,
  megapixels    => 18,
  sensor        => "full-frame 35mm"'
),
( 'MacBook Air 11',
 'manufacturer  => Apple,
  type          => computer,
  ram           => 4GB,
  storage       => 256GB,
  processor     => "1.8 ghz Intel i7 duel core",
  weight        => 2.38lbs'
);
\end{lstlisting}

Теперь можно производить поиск по ключу:
\begin{lstlisting}[label=lst:hstore4,caption=Поиск по ключу]
# SELECT name, attributes->'pages' as page FROM products WHERE attributes ? 'pages';
        name        | page
--------------------+------
 Geek Love: A Novel | 368
(1 row)
\end{lstlisting}

Или по значению ключа:
\begin{lstlisting}[label=lst:hstore5,caption=Поиск по значению ключа]
# SELECT name, attributes->'manufacturer' as manufacturer FROM products WHERE attributes->'type' = 'computer';
       name      | manufacturer
 ----------------+--------------
  MacBook Air 11 | Apple
 (1 row)
\end{lstlisting}

Создание индексов:
\begin{lstlisting}[label=lst:hstore6,caption=Индексы]
CREATE INDEX products_hstore_index ON products USING GIST (attributes);
CREATE INDEX products_hstore_index ON products USING GIN (attributes);
\end{lstlisting}

Можно также cоздавать индекс на ключ:
\begin{lstlisting}[label=lst:hstore7,caption=Индекс на ключ]
CREATE INDEX product_manufacturer
   ON products ((products.attributes->'manufacturer'));
\end{lstlisting}

\subsection{Заключение}

HStore~--- расширение для удобного и индексируемого хранения слабоструктурированых данных в PostgreSQL.