\section{Pg\_repack}
\textbf{Лицензия}: Open Source

Таблицы в PostgreSQL представлены в виде страниц, размером 8Kb, в которых размещены записи. Когда одна страница полностью заполняется записями, к таблице добавляется новая страница. При удалалени записей с помощью DELETE или изменении с помощью UPDATE, место где были старые записи не может быть повторно использовано сразу же. Для этого процесс очистки autovacuum, или команда VACUUM, пробегает по изменённым страницам и помечает такое место как свободное, после чего новые записи могут спокойно записываться в это место. Если autovacuum не справляется, например в результате активного изменения большего количества данных или просто из-за плохих настроек, то к таблице будут излишне добавляться новые страницы по мере поступления новых записей. И даже после того как очистка дойдёт до наших удалённых записей, новые страницы останутся. Получается что таблица становится более разряженной в плане плотности записей. Это и называется эффектом раздувания таблиц, table bloat.

Процедура очистки, autovacuum или VACUUM, может уменьшить размер таблицы убрав полностью пустые страницы, но только при условии что они находятся в самом конце таблицы. Чтобы максимально уменьшить таблицу в PostgreSQL есть VACUUM FULL или CLUSTER, но оба эти способа предполагают установку тяжелых и длительных блокировок на таблицу, что далеко не всегда является подходящим решением.


\subsection{Pgcompact}


\subsection{Заключение}
Pg\_repack~--- расширение для борьбы с <<table bloat>> в PostgreSQL.